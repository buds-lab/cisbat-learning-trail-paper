% !TEX root = 99_main.tex

Cozie is built as a clock-face for fitbit, a wearable health tracker with 25 million active users \cite{fibit2018}. The application is publicly available for download at the following link [insert link]

\subsection{Overview}


SDE Learning Trail describes the NZEB's six different building systems - Energy, Water, Hybrid Cooling, Wellness, Tropical Architecture, and Biophilic Design - as ‘trails’ for occupants and building visitors. Each ‘trail’ is composed of a number of physical “stations”, in the form of QR codes, placed at different areas in the building. A simple web application connects each station with information and interactive visualizations explaining the building systems. This enables occupants and visitors to engage with the new building and appreciate its design, construction, and sustainability features. 
 
As users complete ‘trails’ by visiting different ‘stations’, the application collects user feedback for thermal, visual and aural comfort variables. This data allows the construction of personalized environmental comfort profiles of building occupants and visitors over time - which in turn is used for recommendation of comfortable workspaces. The results of this study provide the proof of concept for SpaceMatch, a spatial recommendation system, which integrates the comfort functionalities of “learning trail”, in order to recommend workspaces that match the comfort profile of the users in real time. 


In this section we define "user" as the test participant who is wearing the fitbit, and "manager" as the person coordinating the experiment. \ 

The default status of the clock-face is a simple binary question: "Comfy" or "Not Comfy", as seen in Figure \ref{fig:homescreen}. By simply clicking one of the icons, information about the users location (GPS), heart-rate, steps walked since last log, and the comfort data is anonymously sent to an Influx time series cloud database [Ref influx]. Data from this database can be simply queried with an API key that can be provided to the manager. Further documentation can be found on the cozie website [insert link].\\

If the manager is interested as to why the user is feeling discomfort, then there is a range of additional questions that can be configured using the cellphone that the fitbit is paired with. The optional questions include: thermal preference, light preference, noise preference, indoor/outdoor, mood, and whether the user is in office. These settings, along with a unique user-id for each user, and a unique experiment-id can be configured by the manager using the cellphone that the fitbit is paired with. The watch-face also has the ability to prompt the user and force them to provde feedback at custom intervals set by the manager.

% \begin{itemize}
%   \item Thermal: Prefer Warmer - Prefer Cooler
%   \item Light: Prefer brighter - Prefer Dimmer
%   \item Noise: Prefer Louder - Prefer Quieter 
%   \item Mood: Good - Neutral - Not So Good
%   \item Location: Indoor - Outdoor
%   \item Location: In Office - Out of Office
% \end{itemize}

%These responses will be grouped with the afore mentioned data, and stored in the Influx time series database. The manager is invited to contact the authors if they have further tailored questions that they would like to add.\\

%The watch-face also has the ability to prompt the user with a 3 second vibration, and force them to provide comfort feedback. This may be triggered at certain hours of the day, random hours of the day, at set time intervals, or at each 1000 steps walked. 

% \begin{figure}
%     \begin{subfigure}[t]{0.3\textwidth}
%         \includegraphics[height= 7cm]{iphone.png}
%     \end{subfigure}
%     \begin{subfigure}[t]{0.3\textwidth}
%         \includegraphics[height= 3cm]{flow.png}
%     \end{subfigure}
%     \caption{Using the fitbit mobile application to design a survey flow}
%     \label{fig:homescreen}
% \end{figure}




% \subsection{Building Data Labeling}

% The human comfort feedback can be combined with building sensor data to create a labeled data set of the environment. (perhaps talk more or delete this section)

% An example of this in practice will be introduced in the next section.