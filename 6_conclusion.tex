% !TEX root = 99_main.tex

This paper describes the pilot implementation of Learning Trail application at the new building in NUS for occupant comfort data collection. Within just three months, 1163 environmental feedback momentary assessment surveys of thermal, visual and aural comfort were obtained participants who gave five or more votes. A total of 616 participants have contributed to the study till date, with minimal administrative overhead. This rich data set provides new opportunities for understanding occupant comfort behavior through data driven methods. Within this study, we've demonstrated how data can be used to group occupants into comfort profile types and shown this study as potential stepping stone to other research areas such as comfort profiling of spaces, occupant behavior analysis and correlation identification between various spatiotemporal variables in buildings. 
Segments of the raw data and analysis code used in this analysis will be available in an open-access Github repository that includes documentation on the Learning Trail app\footnote{\url{https://github.com/buds-lab/cisbat-learning-trail-paper}}. The app is a mobile, web-based platform that can be deployed in other locations in collaboration with the authors.

% As a logical next step, this study provides the proof of concept for the complimentary SpaceMatch project - a spatial recommendation system that utilizes the collected data from this study to match user comfort profiles to spaces that best suit them in real time.

% \subsection{Reproducibility}



