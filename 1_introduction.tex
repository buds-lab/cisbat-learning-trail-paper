% !TEX root = 99_main.tex

%\begin{figure}
%\begin{center}
%\includegraphics[width=\textwidth, trim= 0cm 0cm 0cm 0cm,clip]{cozie-overview.pdf}
%\caption{Overview of the cozie app. (a) the fitbit mobile app is used to set experimental settings, (b) the flow of questions based on settings selected in (a), (c) photos of the watch-face in action with the strap-pack sensor box, (d) overview of data communication.}
%\label{fig:homescreen}
%\end{center}
%\end{figure}

Individual differences in comfort preferences describe the phenomenon that building occupants might perceive their thermal environment differently even when exposed to the same conditions. This essentially means what suits one group or type of occupants may be unacceptable for others - making it challenging to provide comfortable, conditioned environment for everyone in the the same space.

Even though its well understood that differences in individual comfort preferences exist \cite{WANG2018181}, their identification presents significant challenges for researchers and practicioners in the field. Structured surveys or interviews - online or offline, in person or remote - are convectionally used to collect human comfort feedback for buildings. Though these approaches work in principle, they have a number of shortcomings \cite{organisationforeconomicco-operationanddevelopment(oecd)_2013}:  

\begin{itemize}
  \item Subjectivity: There is sufficient evidence to suggest variance in subjective well-being responces based on individual differences in respondent's personality, geographical background and culture \cite{doi:10.1146/annurev.psych.54.101601.145056}
  \item Response bias \& hueristics: A number of factors such as lack of motivation (respondents may not process questions fully), failures in communication (survey questions may be unclear or misunderstood) and lack of knowledge (respondents do not know the answer to a question, but answer it anyway) are often associated with increased risk of biases and respondent hueristics in survey responses \cite{bradburn2004asking}.
  \item Contextual Cueing: Subtle cues in the survey context influence how respondent answer questions \cite{krosnick1997seymour} - since most studies are conducted outside of the respondent's natural working environment - it might have unintended consequences on reponses. 
  \item Scalibility: Its difficult to collect large sample data sets due to the administrative, financial and other overheads of these approaches.
\end{itemize}


This paper presents SDE Learning Trail, a human-building interaction application that enables building visitors and occupants to provide environmental comfort feedback - through their phones - in a simple and intuitive way while helping them learn about the sustainability features of the new Net Zero Energy Building (NZEB) building at the School of Design \& Environment (SDE), National University of Singapore (NUS). The application was launched on Jan 30th 2019 for the new building and has collected [INSERT NUMBER] responses till date. This work demonstrates how the collected occuoant feedback data can used for construction of personalised comfort profiles of occupants and spatial profiles of rooms over time - which in turn provides the proof of concept for SpaceMatch - a spatial recommendation system - that integrates the comfort functionalities of “SDE learning trail” to recommend flexible workspaces that match the comfort profile of the occupants in real time. \\


The remainder of the paper is organised as follows. The next section outlines the SDE Learning Trail application and its deployment at NUS. Section 3 details a how the research team uses the application to collect occupant comfort feedback in the building. Section 4 presents the preliminary results from this deployment till date, and Section 5 discusses our findings and next steps in this project. Finally, Section 6 concludes the paper. 


% \begin{figure}
% \begin{center}
% \includegraphics[width=8cm, trim= 0cm 0cm 0cm 0cm,clip]{facadeFunctionsnew.pdf}
% \caption{The facade acting as a mediator between the interior and exterior environment, while fulfilling various functions \cite{nagy2016adaptive}}
% \label{fig:ASFschematic}
% \end{center}
% \end{figure}

% \begin{figure}
% \begin{center}
% \includegraphics[width=8cm, trim= 0cm 0cm 0cm 0cm,clip]{honr.jpg}
% \caption{An example of an ASF constructed at the House of Natural Resources \cite{nagy2016adaptive}}
% \label{fig:HoNR}
% \end{center}
% \end{figure}




