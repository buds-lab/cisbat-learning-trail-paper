% !TEX root = 99_main.tex

\subsection{Choice of a field based experiment setup}

The goals for conducting comfort assessments under controlled lab settings can be different from conducting the same in field conditions. The former is better suited for dispositional approaches - where the surrounding environment doesn't effect participant behavior, whereas the latter is focused on situational approaches - where behavior is dependent on the surrounding context.
Given that the one aim for this study was to understand the dynamic nature of occupant comfort in different environmental and spatial contexts, the research team chose a field based experiment setup over an artificial setting. This provides higher ecological validity \cite{andrade2018internal}to the findings compared to a lab experiment.           


\subsection{Findings from large data and a 3-point scale}
%\label{ch:localisation}

Generalizing findings for the larger population using detailed surveys or interview results from a small group of participants was a trusted method for comfort assessments in building research in the past. However with new technologies and modern data capabilities, collection, processing and analysis of large datasets has become easier. Thats why this study utilizes QR codes, an interactive mobile application and time-series database infrastructure to collect and process a considerably large comfort assessment data in a short time.\\

As highlighted earlier in section 1, comfort feedback data can be skewed due to a participant's personal traits, geographical and cultural background and response biases. To limit subjectivity and make it easy for participants to provide feedback in field conditions, the team used a 3-point comfort scale rather than the traditional 7-point scale. This not only saved effort, time and hassle for the participants in the field but also the data provided enough insights for the research team to understand comfort preferences of participants.\\

It's interesting to reflect upon the kind of conclusions that can be derived as a result. The traditional method can derive detailed conclusions such as "5 of the 20 users felt comfortable at temperatures lower than 22 $^\circ$C". Whereas the large data method can draw conclusions such as "5 of 20 users can be categorized as a user type that prefers cooler working environments". 



\subsection{Identification of people and space types}

This study identifies personalized comfort profiles of users and spaces in the new building through data driven methods. This detection helps characterize users and spaces into "types" which could in turn be used to understand, and even predict, anomalies in occupant behavior, spatial profiling and building defects in the future.






