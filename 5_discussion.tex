% !TEX root = 99_main.tex

\subsection{Choice of a field based experiment setup}
The goals for conducting comfort assessments under controlled lab settings can be different from conducting the same in field conditions. The former is better suited for dispositional approaches - where the surrounding environment doesn't effect participant behavior, whereas the latter is focused on situational approaches - where behavior is dependent on the surrounding context.
Given that the one of the aims for this study was to understand the dynamic nature of occupant comfort in different environmental and spatial contexts, the research team chose a field based experiment setup to provide higher ecological validity to the findings compared to a lab experiment \cite{andrade2018internal}.           


\subsection{Findings from large data and a 3-point scale}
%\label{ch:localisation}

Generalizing findings for the larger population using detailed surveys or interview results from a small group of participants was a trusted method for comfort assessments in building research in the past. However with new technologies and modern data capabilities, collection, processing and analysis of large data sets has become easier. That's why this study utilizes QR codes, an interactive mobile application and time-series database infrastructure to collect and process a relatively large comfort assessment data set in a short time. As highlighted earlier in Section \ref{ch:introduction}, comfort feedback data can be skewed due to a participant's personal traits, geographical and cultural background and response biases. To limit subjectivity and make it easy for participants to provide feedback in field conditions, the team used a 3-point comfort scale rather than the traditional 7-point comfort scale. This not only saved participant's effort and time in the field but also helped channelize and organize data streams for the research team easily.

% It's interesting to reflect upon the kind of conclusions that can be derived as a result. The traditional method can derive detailed conclusions such as "5 of the 20 users felt comfortable at temperatures lower than 22 $^\circ$C". Whereas the large data method can draw conclusions such as "5 of 20 users can be categorized as a user type that prefers cooler working environments". 


\subsection{Identification of occupant types}

This study identifies personalized comfort profiles of users through data driven methods - which basically cluster users into \emph{types}. This could be used to understand, and even predict, patterns and anomalies in occupant behavior and occupant profiling in the future. Its easy to see how the same methodology could be used to distinguish spaces based on occupant comfort feedback data - to derive comfort profiles of spaces.


\subsection{Limitations}
It is important to note that the new building at NUS - the test bed for this initial pilot implementation of SDE Learning Trail - is operational but still under a defect liability period. Since a majority of the new building's systems are still undergoing calibration and refinement for full building operations in the future, multiple data sources cannot be shared publicly.  









