\documentclass[a4paper]{jpconf} %review, or preprint

\usepackage{amssymb}
\usepackage{graphics}
\usepackage{float}

\usepackage{color}
\usepackage{lscape}

\usepackage[toc,page]{appendix}

%For tracking changes
\usepackage{changes}
\definechangesauthor[name={Tapeesh Sood}, color=orange]{T}
\usepackage[normalem]{ulem}

%Flowcharts
\usepackage[latin1]{inputenc}
\usepackage{tikz}
\usepackage{circuitikz}
\usetikzlibrary{shapes,arrows}
\tikzstyle{decision} = [diamond, draw, 
    text width=4.5em, text badly centered, node distance=3cm, inner sep=0pt]
\tikzstyle{block} = [rectangle, draw, 
    text width=6em, text centered, rounded corners, minimum height=3em]
\tikzstyle{line} = [draw, -latex']
\tikzstyle{cloud} = [draw, ellipse, node distance=3cm,
    minimum height=2em]
\tikzstyle{noborder} = [node distance=3cm, text width=6em, text centered, minimum height=2em]

%For subfigure
\usepackage{graphicx}
\usepackage{caption}
\usepackage{subcaption}

%Stop footnote from skipping to next page
\interfootnotelinepenalty=10000

%Some code to display the units after the equation 
\usepackage{mathtools}
\makeatletter
\providecommand\add@text{}
\newcommand\tagaddtext[1]{%
  \gdef\add@text{#1\gdef\add@text{}}}% 
\renewcommand\tagform@[1]{%
  \maketag@@@{\llap{\add@text\quad}(\ignorespaces#1\unskip\@@italiccorr)}%
}
\makeatother



\graphicspath{{./Images/}}



%\journal{Solar Energy, Energy and Buildings, Building and Environment, Energy Science and Engineering}

\begin{document}

%\begin{frontmatter}

% \begin{figure}
% \begin{center}
% \includegraphics[width=\columnwidth, trim= 0cm 0cm 0cm 0cm,clip]{HeaderEP.pdf}
% \label{fig:header}
% \end{center}
% \end{figure}

% \begin{center}
% {CISBAT 2019 International Conference - Future Buildings \& Districts - Energy Efficiency from Nano to Urban Scale, Lausanne, Switzerland}
% \end{center}

\title{Crowdsourcing occupant comfort feedback at a net zero energy building in the tropics} 
%Energy Performance of PV modules as Adaptive Building Shading Systems
%Numerical Energy Analysis of PV Modules as Adaptive Building Shading Systems

\author{ T Sood$^1$, M Quintana$^1$,
P Jayathissa$^1$, M. AbdelRehman$^1$, C. Miller$^1$ \footnote[4]{Present address:
Department of Building, National University of Singapore, 4 Architecture Drive, Singapore}}
\address{$^1$ Building and Urban Data Science Group,  Department of Building, Singapore}
\ead{tapeeshsood@nus.edu.sg}



% \author[buds]{T Sood\corref{cor2}}
%     \ead{tapeeshsood@nus.edu.sg}
% \address[buds]{Building and Urban Data Science Group,  Department of Building, Singapore} 
% % For whatever reason the affiliation needs to be defined after the authors. Otherwise the numbering gets messed up.

% \author[buds]{M. Quintana}
%   \ead{matias@u.nus.edu}

% \author[buds]{P. Jayathissa}
%   \ead{p.jayathissa@u.nus.edu}

% \author[unsw]{M.AbdelRehman}
%     \ead{mahmoud@u.nus.edu}

% \author[buds]{C. Miller \corref{cor1} }
%     \ead{clayton@nus.edu.sg}


%\cortext[cor2]{Corresponding author}


\begin{abstract}



This study shares a human-building interaction framework - SDE Learning Trail - currently deployed at the new Net Zero Energy Building (NZEB), School of Design \& Environment (SDE), National University of Singapore. The framework enables building visitors to learn about well & green features of the new building while collecting environmental comfort feedback in a simple and intuitive way. This paper provides    

[TAPEESH: I left both versions of the text for you to choose which one you want]

This study shares the development of a human-building interaction application - SDE Learning Trail - currently deployed at the new Net Zero Energy Building (NZEB), School of Design \& Environment (SDE), National University of Singapore. The application enables building visitors to learn about the sustainability features of the new building while collecting environmental comfort feedback in a simple and intuitive way. Since its launch three months ago, the application has collected [insert number] responses from [insert number] users. This work also demonstrates how the collected feedback data is used for construction of personalised comfort profiles of users and spatial profiles of spaces over time. The results of this study provide the proof of concept for SpaceMatch - a spatial recommendation system - which integrates the comfort functionalities of "SDE learning trail", in order to recommend flexible workspaces that match the comfort profile of the users in real time.    


 
[TAPEESH: I left both versions of the text for you to choose which one you want]


This study shares a human-building interaction framework - SDE Learning Trail - currently deployed at the new Net Zero Energy Building (NZEB), School of Design \& Environment (SDE), National University of Singapore. The framework enables building visitors to learn about well & green features of the new building while collecting environmental comfort feedback in a simple and intuitive way. Since its launch three months ago, the application has collected [insert number] responses from [insert number] users. This paper covers how the collected feedback data is being used for construction of personalised comfort profiles of users and spatial profiles of building rooms over time. The results of this study provide the proof of concept for SpaceMatch - a spatial recommendation system - that helps match comfortable flexible workspaces to comfort profiles of users in real time.


\end{abstract}

% \begin{keyword}
% Comfort Feedback \sep Data Collection \sep Fitbit \sep Comfort Recommendation \sep Mood Logging
% \end{keyword}

% \end{frontmatter}

\section{Introduction}
\label{ch:introduction}
% !TEX root = 99_main.tex

Filler Textjbjjkh

Preliminary results have already been obtained and successfully mapped to wearable enviornmn



% The state of the art in comfort data collection is through surveys, where the test subjec

% Moving away from a one size fits all model is possible in modern flexible work spaces. 

% An alternative approach in improving the comfort of a room 

% This paper presents an alternative approach to improoving human comfort tailoring where the occupant's comfort preference is matched to the comfort properties of a room.



\begin{figure}
\begin{center}
\includegraphics[width=0.5\textwidth, trim= 0cm 0cm 0cm 0cm,clip]{strap-pack.jpg}
\caption{The facade acting as a mediator between the interior and exterior environment, while fulfilling various functions \cite{IPCC}}
\label{fig:ASFschematic}
\end{center}
\end{figure}

% \begin{figure}
% \begin{center}
% \includegraphics[width=8cm, trim= 0cm 0cm 0cm 0cm,clip]{honr.jpg}
% \caption{An example of an ASF constructed at the House of Natural Resources \cite{nagy2016adaptive}}
% \label{fig:HoNR}
% \end{center}
% \end{figure}






\section{SDE Learning Trail}
\label{ch:SDELT}
% !TEX root = 99_main.tex


% \subsection{Overview}

% Please note from this section onward we address occupant or visitor to the new NZEB as ``user'' or ``participant'', our team coordinating the experiment as ``research team'', and the new NZEB as ``building''.\\ 






% \begin{itemize}
%   \item Thermal: Prefer Warmer - Prefer Cooler
%   \item Light: Prefer brighter - Prefer Dimmer
%   \item Noise: Prefer Louder - Prefer Quieter 
%   \item Mood: Good - Neutral - Not So Good
%   \item Location: Indoor - Outdoor
%   \item Location: In Office - Out of Office
% \end{itemize}

%These responses will be grouped with the afore mentioned data, and stored in the Influx time series database. The manager is invited to contact the authors if they have further tailored questions that they would like to add.\\

%The watch-face also has the ability to prompt the user with a 3 second vibration, and force them to provide comfort feedback. This may be triggered at certain hours of the day, random hours of the day, at set time intervals, or at each 1000 steps walked. 

% \begin{figure}
%     \begin{subfigure}[t]{0.3\textwidth}
%         \includegraphics[height= 7cm]{iphone.png}
%     \end{subfigure}
%     \begin{subfigure}[t]{0.3\textwidth}
%         \includegraphics[height= 3cm]{flow.png}
%     \end{subfigure}
%     \caption{Using the fitbit mobile application to design a survey flow}
%     \label{fig:homescreen}
% \end{figure}




% \subsection{Building Data Labeling}

% The human comfort feedback can be combined with building sensor data to create a labeled data set of the environment. (perhaps talk more or delete this section)

% An example of this in practice will be introduced in the next section.

\section{Methodology}
\label{ch:method}
% !TEX root = 99_main.tex

An experient was conducted at co-working spaces at the National Unviersity of Singapore. 15 participants were recruited for the experiment and equiped with a fitbit watch. The watch settings were set to also request thermal preference (prefer warmer, prefer cooler, comfy), and the set to force request feedback at the hours of 9:00, 11:00, 13:00, 15:00, and 17:00 

The watch was further complimented with IoT connected on-body and environmental sensors. The onbody sensor consists of a temperature and light sensor from mbient-labs that had been modified to fit the watch strap with a custom 3d printed case. An off body sensor measuring temperature and humidity was attached to the participants bag. The sensors communicate via bluetooth to raspberryPi gateways that had been positioned throughout the working space. 

Data from the cozie watch face, and the environmental sensors were aggrigated in an Influx cloud time-series database, which served as a platform for data aquisition and fault detection. Source codes can be found here [ref aurek-data-crunch repo].


% \begin{figure}
% \begin{center}
% \includegraphics[width=0.2\textwidth, trim= 0cm 0cm 0cm 0cm,clip]{strap-pack.jpg}
% \caption{Strap-Pack}
% \label{fig:strappack}
% \end{center}
% \end{figure}



\section{Results}
\label{ch:results}
% !TEX root = 99_main.tex



\subsection{Overview of the feedback data}

Figure \ref{fig:feedbackdata}a shows the distribution of feedback votes between indoor and outdoor spaces. It's interesting to note that most feedback votes have been collected at outdoor stations in the trail. Limited nudging of users to provide feedback to avoid feedback fatigue \cite{EffectsFeedbackFatigue} and controlled access to participants for stations within indoor spaces are some of the important reasons for this distribution.\\

Further the locations for outdoor stations are naturally ventilated but shaded due to the building's large, overarching roof and open design. These environmental conditions had an effect on the feedback received through the application as shown in Figure \ref{fig:feedbackdata}b.  As shown, the figure provides a summary of all feedback vote categories for Temperature, light and noise variables. Its important to note that certain categories were less favored than others due to a majority feedback gathered from stations located outdoor. For instance, "Prefer Warmer", "Prefer Louder" responses were considerably less employed by the participants as compared to others in the same category due to external weather and urban conditions of the building site.\\    


\begin{figure}
\begin{center}
\includegraphics[width=\textwidth, trim= 0cm 0cm 0cm 0cm,clip]{Fig3.jpg}
\caption{Overview of Feedback Data: (a) Split between outdoor vs. indoor feedback data (b) Distribution of votes between temperature, light and noise variables.}
\label{fig:feedbackdata}
\end{center}
\end{figure}


\subsection{Distinguishing spaces based on comfort profiles}
\label{ch:userResults}

Individual user feedback was clustered using un-supervised learning techniques. We used Ward's method for hierarchal clustering based on standardized euclidean distance.\\
The results, shown in Figure \ref{fig:clustering}a, show 8 distinct clusters for comfort profiles of spaces based on user feedback. It can be observed that spaces are most frequently perceived as "comfy" or comfortable followed by preferences for more cooling and less noise. This pattern is reflective of majority of the feedback collected from outdoor spaces as noted before.\\

It is important to recognize that unique profiles for spaces can be identified based on user's comfort perception of temperature, light and noise conditions. For instance, space cluster "A" is 40-50\% of the time thermally and aurally comfortable and 80\% or more visually comfortable. On the other hand, space cluster "F" is perceived 80-100\% thermally, aurally, visually comfortable most times.           


\subsection{Discovering occupant comfort profiles}
 
As shown in Figure \ref{fig:clustering}b, distinct [INSERT NUMBER] clusters can be observed based on differences between preferences for temperature, light and noise levels across users. Generally users are comfortable most times even in outdoor spaces, but could prefer cooler and quieter surroundings.\\

Further, its interesting to discover differences in user "types" based on the clustering. For instance, type "A" users were 70-85\% of the time visually and thermally comfortable but only 30-45\% of the time aurally comfortable in the building. On the other hand, type "R" users were 85-100\% visually and aurally comfortable but 35-50\% thermally comfortable. Understanding and defining these differences between user types can be used to personalize spatial recommendations to individual users based on their past preferences.       

Its important to note that we slightly changed our approach for clustering individual user preferences for this analysis - though we used Ward's method for hierarchal clustering but changed to euclidean distance.\\    


\begin{figure}
\begin{center}
\includegraphics[width=\textwidth, trim= 0cm 0cm 0cm 0cm,clip]{Fig4.jpg}
\caption{Clustering: (a) Clustering space type based on user feedback (b) Clustering users based on comfort preferences.}
\label{fig:clustering}
\end{center}
\end{figure}




\subsection{Correlations between environemntal attributes in different types of spaces}

%Combining the cozie watch face, with the "strap-pack", an environmental sensor addition to the watch face opens another dimension of analysis. User responses are mapped to the environmental condition at which they are exposed to, which can provide a high quality labeled data set for training data driven models. Figure \ref{fig:tempHist} detail the temperatures at which responses were mapped. Note that the temperature of the strap sensor is on average 0.8 $^\circ$C warmer than the surrounding environment due to the influence of body temperature. 

\begin{figure}
\begin{center}
\includegraphics[width=\textwidth, trim= 0cm 0cm 0cm 0cm,clip]{Fig5.jpg}
\caption{Correlations (a) Overall (b) Indoor spaces, (c) Outdoor spaces.}
\label{fig:Clustering}
\end{center}
\end{figure}



% convert time to bars
% one heart rate filter showing. Groups of similar behaving people. Group 1-4. What are the coincidental ranges of data belonging to these groups. 


\section{Discussion}
\label{ch:discussion}
% !TEX root = 99_main.tex

\subsection{Choice of a field based experiment setup}
The goals for conducting comfort assessments under controlled lab settings can be different from conducting the same in field conditions. The former is better suited for dispositional approaches - where the surrounding environment doesn't effect participant behavior, whereas the latter is focused on situational approaches - where behavior is dependent on the surrounding context.
Given that the one of the aims for this study was to understand the dynamic nature of occupant comfort in different environmental and spatial contexts, the research team chose a field based experiment setup to provide higher ecological validity to the findings compared to a lab experiment \cite{andrade2018internal}.           


\subsection{Findings from large data and a 3-point scale}
%\label{ch:localisation}

Generalizing findings for the larger population using detailed surveys or interview results from a small group of participants was a trusted method for comfort assessments in building research in the past. However with new technologies and modern data capabilities, collection, processing and analysis of large data sets has become easier. That's why this study utilizes QR codes, an interactive mobile application and time-series database infrastructure to collect and process a relatively large comfort assessment data set in a short time. As highlighted earlier in Section \ref{ch:introduction}, comfort feedback data can be skewed due to a participant's personal traits, geographical and cultural background and response biases. To limit subjectivity and make it easy for participants to provide feedback in field conditions, the team used a 3-point comfort scale rather than the traditional 7-point comfort scale. This not only saved participant's effort and time in the field but also helped channelize and organize data streams for the research team easily.

% It's interesting to reflect upon the kind of conclusions that can be derived as a result. The traditional method can derive detailed conclusions such as "5 of the 20 users felt comfortable at temperatures lower than 22 $^\circ$C". Whereas the large data method can draw conclusions such as "5 of 20 users can be categorized as a user type that prefers cooler working environments". 


\subsection{Identification of occupant types}

This study identifies personalized comfort profiles of users through data driven methods - which basically cluster users into \emph{types}. This could be used to understand, and even predict, patterns and anomalies in occupant behavior and occupant profiling in the future. Its easy to see how the same methodology could be used to distinguish spaces based on occupant comfort feedback data - to derive comfort profiles of spaces.


\subsection{Limitations}
It is important to note that the SDE4 building is operational, but still under a defect and liability period. Since a majority of the new building's systems are still undergoing calibration and refinenement for full operations in the future, multiple data sources are not integrated into this analysis. This analysis is preliminary and further convergence of data from these other sources such as fixed environmental sensors, demographic information, and physiological information from wearable devices will produce more traditional comfort modeling insight. Additionally, given nature of use of this application for orientation tours of SDE4, users are most likely to participate only once. While this constraints the comparison of occupant comfort feedback and spatio-environmental variables over time for each participant - at the end of the tour, participants are pointed to an advanced spatial recommendation application which uses their comfort feedback to match them to suitable spaces in the building for flexible working. More nuanced occupant feedback and spatial-environmental data is collected for participants which use these flexible workspaces over time. More details regarding the spatial recommendation system and its application would be provided in a future publication.











\section{Conclusion}
\label{ch:conclusion}
% !TEX root = 99_main.tex

First pilot implementation of Learning Trail application at the new building in NUS for occupant comfort data collection has proven successful. Within just 3 months, 1163 data points of thermal, visual and aural comfort were obtained from the 616 participants, with minimal administrative overhead.\\

This rich data set provides new opportunities for analyzing occupant comfort behavior through data driven methods. Within this study, we've demonstrated how data can be used to group occupants and spaces based on comfort profile types. For the new building at NUS, we identified [INSERT NUMEBR HERE] distinct user types and [INSERT NUMEBR HERE] distinct space types.\\

As next steps this study provides the proof of concept for SpaceMatch - a spatial recommendation system - that utilizes the user types to match space profiles to the user comfort profiles in real time through an easy to use mobile application.
\\

% \section{Acknowledgments}
% \label{ch:acknowledgments}
% \input{7_acknowledgments}

% \begin{appendices}
%  \label{ch:appendix}
%  \input{8_appendix}
%  \end{appendices}

%% appendix sections are then done as normal sections
%% \appendix
%% \section{}
%% \label{}

%% bibitems, please use
\section*{References}
  \bibliographystyle{elsarticle-num} 
  \bibliography{references}

\end{document}
\endinput
